\graphicspath{{images/university/}}
\section{Об университете}
	Раздел позволяет просматривать и редактировать информацию, связанную с университетом: основную информацию, контакты, а также медиа"=файлы, отвечающие за отображение страниц, связанных с этим вузом на сайте.
	
	\subsection{Роли и операции}
	
	Раздел доступен пользователям, имеющим следующие роли:	

	\begin{itemize}
		\item Администратор Платформы:
		\begin{itemize}
			\item просмотр информации об университете;
			\item редактирование информации об университете;
			\item создание нового университета.
		\end{itemize}
		\item Администратор вуза"=поставщика:
		\begin{itemize}
			\item просмотр информации об университете;
			\item редактирование информации об университете.
		\end{itemize}
		\item Администратор контента вуза"=поставщика:
		\begin{itemize}
			\item просмотр информации об университете;
			\item редактирование информации об университете.
		\end{itemize}
	\end{itemize} 
	
	\subsection{Просмотр и редактирование информации об университете}\label{university:detail_section}
	При выборе в главном меню пункта \quotes{Об университете} загружается подраздел \quotes{Просмотр информации}, в котором доступны для просмотра все имеющиеся на текущий момент данные об университете. Для перехода к подразделу редактирования необходимо нажать на иконку \quotes{редактирование} \vcenteredinclude[width=0.1\linewidth]{edit_icon.png} в верхнем правом углу страницы подраздела \quotes{Просмотр информации}.

	\subsubsection{Поля и ошибки}
	В форме представлено несколько типов полей:
	\begin{itemize}
		\item \textbf{Обязательные поля} отмечены символом \quotes{*}, если такое поле оставить не заполненным "--- появляется сообщение о необходимости его заполнения и блокируется кнопка сохранения изменений.
	
		\item \textbf{Сайт и адрес электронной почты университета}. Ввод информации в эти поля необходимо осуществлять в корректном формате, например, \texttt{http://npoed.ru} для сайта и \texttt{npoed@mail.ru} для почты соответственно, в противном случае появится сообщение об ошибке и будет заблокирована кнопка сохранения изменений.

		\item \textbf{Поле \quotes{порядок преподавателей}} представляет собой список ФИО преподавателей университета в том порядке, в котором они будут показаны на сайте. Описание работы с виджетом изменения порядка см. в подразделе~\ref{widget:ordering}

		Для каждого преподавателя отдельно можно настроить его видимость на сайте в подразделе редактирование преподавателя путем переключения значения поля \quotes{Опубликован}, неопубликованные преподаватели скрыты на сайте для пользователей. В форме редактирования университета в списке преподавателей показываются как скрытые, так и опубликованные преподаватели. В зависимости от статуса их ФИО имеют различный цвет. Подробное описание можно получить, нажав на иконку \quotes{помощь} рядом с меткой поля.
		
		\item \textbf{Файлы}. Форма редактирования позволяет загружать изображения для формирования внешнего вида страницы университета на сайте. 
\end{itemize}
	
	\subsubsection{Предпросмотр}\label{university:edit_preview_ch}
	Для подраздела \quotes{Редактирование университета} доступна опция предпросмотра, с помощью которой можно оценить внешний вид страницы университета на сайте для пользователей с учётом внесенных изменений. Предпросмотр можно осуществить даже в том случае, если форма заполнена некорректно или не полностью. Страница загружается во всплывающем окне при нажатии кнопки \quotes{Предпросмотр} внизу страницы с формой редактирования. При этом все внешние ссылки на отображаемой странице заблокированы.
		
\subsection{Создание университета}
Возможность создания кабинета университета доступна администратору Платформы. 

Для перехода в подраздел необходимо выбрать в меню \quotes{Мой профиль}, расположенном в верхнем правом углу страницы, пункт \quotes{Добавить вуз}. В этом случае осуществляется переход к форме создания университета. Содержание, возможные действия и реакции в этом разделе аналогичны описанным в разделе \quotes{Редактирование информации об университете}~\ref{university:edit} за исключением двух моментов:
\begin{itemize}
	\item В форме создания задается уникальный код университета на Платформе (поле \quotes{Код}), который в последствие не может быть изменен. Необходимым условием валидности кода является его уникальность, при наборе уже существующего на Платформе значения появится сообщение об ошибке.
		
	\item Вторым отличием от формы редактирования является отсутствие списка преподавателей. Они добавляются после создания университета через раздел \quotes{Преподаватели}, и после добавления отображаются в разделах \quotes{Просмотр информации об университете}, \quotes{Редактирование университета}, \quotes{Преподаватели}.
\end{itemize}
	
	

		

	
	
	
	
	
	