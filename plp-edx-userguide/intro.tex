\section{Область применения}
Автоматизированное рабочее место представителя вуза (далее АРМ вуза) является частью Платформы 
\quotes{Открытое образование} (\url{https://openedu.ru}) и предназначено для автоматизации действий, 
связанных с процессами создания, запуска и обеспечения прохождения студентами курсов (в т.ч. "--- получение статистики
действий студентов), зачисления студентов на курсы, контроля работы представителей вуза в рамках Платформы, 
представления вуза на Платформе.

\section{Перечень эксплуатационной документации}

Перечень эксплуатационных документов, с которым необходимо ознакомиться:
\begin{itemize}
	\item АРМ вуза \quotes{Частное техническое задание};
	\item АРМ вуза \quotes{Руководство пользователя};
	\item АРМ вуза \quotes{Руководство системного администратора}.
\end{itemize}

\section{Сокращения и термины}

\begin{itemize}
	\item Платформа, Система "--- Национальная Платформа открытого образования (НПОО) "--- автоматизированная система, 
	развернутая на серверах НПОО и доступная в Интернет по адресам: \\ 
	\url{https://openedu.ru}, \texttt{\href{https://openedu.ru}{открытоеобразование.рф}}.
	\item Вуз"=разработчик, вуз"=поставщик "--- образовательная организация, имеющая лицензию на реализацию образовательных программ 
	высшего образования, осуществившая разработку открытого онлайн курса, размещенного на портале \quotes{Открытое образование}
	и реализующая образовательные программы дополнительного образования на основе этого курса.
	\item Вуз"=потребитель "--- образовательная организация, имеющая лицензию на реализацию образовательных программ 
	высшего образования, заключившая договор с Ассоциацией и сетевой договор с вузом"=разработчиком курса.
	\item Сессия курса "--- конкретный запуск курса на Платформе, имеющий дату начала и окончания, 
	создающий собственное пространство для общения и взаимодействия студентов и команды курса.
	\item Режим прохождения "--- условия доступа студента к содержимому и заданиям онлайн курса. 
	На Платформе для сессии курса могут быть доступны три режима прохождения:
	\begin{itemize}
		\item прослушивание (аналог в edX "--- audit);
		\item прохождение с сертификатом без подтверждения личности (аналог в edX "--- honor);
		\item прохождение в режиме подтверждения личности (аналог в edX "--- verified).
	\end{itemize}
\end{itemize}


