\section{Описание ролей} \label{sec:role_description}
Перечень ролей и набор операций, доступный для выполнения в АРМ для пользователей данных ролей, приведён в списке ниже. Каждая роль связана с типом параметра. Возможные типы параметров: вуз, курс. При назначении роли пользователю выбирается значение параметра: один из вузов или курсов. Одному пользователю можно назначить неограниченное количество ролей с разными значениями параметров. Например, один пользователь может быть администратором нескольких вузов и/или автором нескольких курсов.\\

Набор отображаемых разделов и доступных пунктов меню (операций) формируется динамически на основе полномочий пользователя "--- пользователю будут доступны только те разделы и пункты меню, которые соответствуют его роли.\\

Перечень доступных разделов и операций для каждой роли приведён в следующем списке.

\begin{enumerate}
	\item Администратор вуза\\
	Тип параметра: вуз. При назначении пользователю роли необходимо выбрать вуз, администратором которого является пользователь. Роль может быть назначена для вузов, которые являются поставщиками и потребителями, и для вузов, которые являются только потребителями.
	\begin{itemize}

		\item Если вуз является только потребителем:
		\begin{itemize}
			\item Раздел \quotes{Об университете}
			\begin{itemize}
				\item просмотр и редактирование (кроме флага «Является только потребителем») информации об университете;
			\end{itemize}			
			\item Раздел \quotes{Студенты}
			\begin{itemize}
				\item загрузка студентов списком;
				\item просмотр списка студентов вуза;
				\item просмотр подробной информации о студенте вуза;
				\item создание, редактирование, просмотр заявки на зачисление;
				\item создание, редактирование, просмотр заявки на изменение режима прохождения сессии курса.
			\end{itemize}
			\item Раздел \quotes{Аналитика}
			\begin{itemize}
				\item просмотр аналитических сведений о прохождении курсов студентами своего вуза.
			\end{itemize}
			\item Раздел \quotes{Договоры}
			\begin{itemize}
				\item просмотр списка и подробной информации о договорах на получение услуг.
			\end{itemize}			
		\end{itemize}

		\item Если вуз является поставщиком и потребителем, то помимо операций, которые доступны для потребителя, пользователю с ролью \quotes{Администратор вуза} доступы следующие операции:
		\begin{itemize}
			\item Раздел \quotes{Преподаватели}
			\begin{itemize}
				\item просмотр списка преподавателей, создание, просмотр профиля, редактирование и удаление преподавателя.
			\end{itemize}			
			\item Раздел \quotes{Курсы}
			\begin{itemize}
				\item просмотр списка курсов, создание, просмотр карточки, редактирование и удаление всех курсов вуза;
				\item назначение авторов всех курсов вуза;
				\item просмотр списка сессий курсов, создание, просмотр карточки, редактирование, перезапуск и удаление всех сессий курсов вуза;
			\end{itemize}
			\item Раздел \quotes{Сотрудники}
			\begin{itemize}
				\item просмотр списка сотрудников, профиля сотрудника, приглашение сотрудников на Платформу, просмотр журнала действий сотрудников, назначение пользователям ролей на данный вуз и курсы вуза.
			\end{itemize}
			\item Раздел \quotes{Студенты}
			\begin{itemize}
				\item просмотр студентов, обучающихся на курсах вуза;
				\item зачисление студентов;
				\item отчисление студентов;
				\item изменение режима прохождения сессии курса студентами;
			\end{itemize}
			\item Раздел \quotes{Аналитика}
			\begin{itemize}
				\item просмотр аналитических сведений о прохождении курсов вуза.
			\end{itemize}
			\item Раздел \quotes{Договоры}
			\begin{itemize}
				\item создание просмотр списка и подробной информации, редактирование договора на предоставление услуг;
				\item просмотр списка платежей;
				\item просмотр заявок на зачисление и управление статусом заявок;
				\item просмотр заявок на изменение режима прохождения сессий и управление статусом заявок.
			\end{itemize}			
		\end{itemize}

	\end{itemize}
	\item Администратор контента вуза\\
	Тип параметра: вуз. При назначении пользователю роли необходимо выбрать вуз, администратором контента которого является пользователь. Роль может быть назначена только для вузов, которые являются поставщиками и потребителями.
	\begin{itemize}
		\item Раздел \quotes{Об университете}
		\begin{itemize}
			\item просмотр и редактирование (кроме флага «Является только потребителем») информации об университете;
		\end{itemize}			
		\item Раздел \quotes{Преподаватели}
		\begin{itemize}
			\item просмотр списка преподавателей, создание, просмотр профиля, редактирование и удаление преподавателя.
		\end{itemize}			
		\item Раздел \quotes{Курсы}
		\begin{itemize}
			\item просмотр списка курсов, просмотр карточки, редактирование  всех курсов вуза;
			\item просмотр списка сессий курсов, просмотр карточки, редактирование всех сессий курсов вуза;
		\end{itemize}
	\end{itemize}
	
	\item Автор курса\\
	Тип параметра: курс. При назначении пользователю роли необходимо выбрать курс, автором которого является пользователь. Роль может быть назначена только для вузов, которые являются поставщиками и потребителями.
	\begin{itemize}
		\item Раздел \quotes{Об университете}
		\begin{itemize}
			\item просмотр информации об университете;
		\end{itemize}			
		\item Раздел \quotes{Курсы}
		\begin{itemize}
			\item просмотр списка курсов, просмотр карточки, редактирование курсов, для которых пользователь является автором;
			\item назначение авторов курсов, для которых пользователь является автором;
			\item просмотр списка сессий курсов, создание, просмотр карточки, редактирование перезапуск и удаление всех сессий курсов, для которых пользователь является автором;
		\end{itemize}
		\item Раздел \quotes{Студенты}
		\begin{itemize}
			\item просмотр студентов, обучающихся на курсах, автором которых является пользователь.
		\end{itemize}
		\item Раздел \quotes{Аналитика}
		\begin{itemize}
			\item просмотр аналитических сведений о прохождении курсов, автором которых является пользователь.
		\end{itemize}
		\item Раздел \quotes{Договоры}
		\begin{itemize}
			\item просмотр списка и подробной информации о договорах на предоставление услуг (выводится информация только по тем договорам, в которых участвуют курсы, автором которых является данный пользователь).
		\end{itemize}					
	\end{itemize}

\end{enumerate}

Помимо перечисленных ролей в АРМ доступна роль \quotes{Суперпользователь}. Она предназначена для сотрудников поддержки Платформы. Помимо операций, перечисленных в списке выше, для пользователей данной роли доступны следующие:

\begin{itemize}
	\item Меню пользователя: создание нового университета;
	\item Раздел \quotes{Об университете}: редактирование любого вуза, в т.ч. флага \quotes{Является только потребителем};
	\item Меню пользователя: назначение ролей (необходимо для назначения администраторов вузов).
\end{itemize}
