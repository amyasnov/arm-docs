\section{Установка PLP}
Действия по установке необходимо выполнять от пользователя без прав администратора, кроме пунктов, где отдельно указано обратное. Для установки \texttt{PLP} необходимо выполнить следующие действия:
\begin{itemize}
	\item Скопировать исходный код \texttt{PLP} из репозитория \url{https://github.com/npoed/npoed-plp-edx} в какой-либо каталог (создание каталога для файлов подсистемы может потребовать выполнения от администратора системы) и перейти в этот каталог в интерпретаторе командной строки.
	\item Установить зависимости из файла apt-requrements.txt (от администратора системы). Названия пакетов в нём приводятся для ОС Ubuntu Linux и могут отличаться в других ОС.
	\item Создать виртуальное окружение Python командой:\\
		\texttt{virtualenv .env}
	\item Активировать созданное виртуальное окружение командой:\\
		\texttt{source .env/bin/activate}
	\item Установить Python-зависимости командой:\\
		\texttt{pip install -r requirements.txt}
\end{itemize}
\section{Настройка RabbitMQ и Celery}
Некоторые функции \texttt{PLP} и \texttt{SSO} выполняются асинхронно с помощью \texttt{Celery}. Поэтому для полноценного их функционирования нужно
настроить \texttt{Celery} и \texttt{RabbitMQ}, который выступает в качестве менеджера очереди сообщений для \texttt{Celery}.

Для настройки \texttt{RabbitMQ} нужно выполнить следующие действия:
\begin{itemize}
	\item Создать виртуальный хост\\
		\texttt{sudo rabbitmqctl add\_vhost <имя хоста>}
	\item Создать пользователя\\
		\texttt{sudo rabbitmqctl add\_user <имя пользователя> <пароль>}
	\item Выдать пользователю права на созданный хост\\
		\texttt{sudo rabbitmqctl set\_permissions -p <имя хоста> <имя пользователя> ".*" ".*" ".*"}
\end{itemize}

Даже если \texttt{PLP} и \texttt{SSO} развёрнуты на одном компьютере, хосты в \texttt{RabbitMQ} для них должны быть созданы разные.

Для \texttt{Celery} нужно настроить только автоматический запуск при старте системы. Для этого нужно найти в дистрибутиве \texttt{Celery}
скрипт соответствующий предпочитаемому в Вашей ОС способу запуска (\texttt{init-script}, \texttt{systemd}, \texttt{supervisord} и т.д.)
и скопировать его в соответствующее место. Кроме этого нужно создать конфигурационный файл \texttt{/etc/default/celeryd} следующего содержания
для \texttt{PLP}:
\lstinputlisting[style=text]{src/celeryd_plp}
и для \texttt{SSO}:
\lstinputlisting[style=text]{src/celeryd_sso}

\section{Настройка SSO}
\begin{itemize}
	\item В настройках \texttt{SSO} (\texttt{npoed\_sso\_edx/local\_settings.py}) нужно задать:
	\begin{itemize}
		\item \texttt{PLP\_URL} --- адрес, по которому будет развёрнуто \texttt{PLP}
		\item \texttt{SSO\_API\_KEY} --- \texttt{API}-ключ \texttt{SSO} (задать, если ещё не задан) (произвольная строка)
		\item \texttt{BROKER\_VHOST} --- имя виртуального хоста RabbitMQ
		\item \texttt{BROKER\_USER} --- имя пользователя RabbitMQ
		\item \texttt{BROKER\_PASSWORD} --- пароль пользователя RabbitMQ
	\end{itemize}
	\item В административной панели нужно внести следующие изменения:
	\begin{itemize}
		\item В разделе \texttt{Provider/Clients} нужно создать клиента со следующими параметрами:
		\begin{itemize}
			\item \texttt{Url} --- адрес, по которому будет развёрнут \texttt{PLP}
			\item \texttt{Redirect uri} --- адрес, по которому будет развёрнут \texttt{PLP} + \texttt{/complete/npoedsso/}
			\item \texttt{Client id} и \texttt{Client secret} --- эти поля заполнятся автоматически,
				их значения нужно будет перенести в настройки \texttt{PLP}
		\end{itemize}
		\item В разделе \texttt{Oauth2\_Provider/Trusted clients} нужно создать доверенного клиента для только созданного клиента \texttt{PLP}
		\item В разделе \texttt{Authtoken/Tokens} нужно создать токен (если он ещё не создан) для пользователя, имеющего набор прав суперпользователя
			(право \texttt{*/*/*})
	\end{itemize}
\end{itemize}

\section{Настройка EDX}
\subsection{Настройка параметров взаимодействия с PLP}
\begin{itemize}
	\item В конфигурационных файлах \texttt{cms.env.json} и \texttt{lms.env.json} необходимо задать параметр \texttt{PLP\_URL} ---
		адрес, по которому будет развёрнуто \texttt{PLP}
	\item Необходимо установить расширение \texttt{open\_edx\_api\_extension\_cms}. Для этого:
	\begin{itemize}
		\item Необходимо установить расширение с помощью \texttt{pip}
		\item В файле \texttt{cms/envs/common.py} в параметр \texttt{INSTALLED\_APPS} нужно добавить \texttt{\textquotesingle{}open\_edx\_api\_extension\_cms\textquotesingle{}}
		\item В файле \texttt{cms/urls.py} в параметр \texttt{urlpatterns} нужно добавить
			\texttt{url(r\textquotesingle{}\^{}api/extended/\textquotesingle{}, include(\textquotesingle{}open\_edx\_api\_extension\_cms.urls\textquotesingle{}, namespace=\textquotesingle{}api\_extension\textquotesingle{}))}
		\item В файле \texttt{cms/envs/aws.py} нужно добавить строку \texttt{EDX\_API\_KEY = AUTH\_TOKENS.get("EDX\_API\_KEY")}
		\item В кофигурационном файле \texttt{cms.auth.json} нужно задать параметр \texttt{EDX\_API\_KEY},
			причем таким же значением, как и в файле \texttt{lms.auth.json}
		\item Если в \texttt{EDX CMS} отсутствует пользователь с ролью \texttt{GlobalAdmin} (пользователь с \texttt{is\_staff} равным \texttt{True}),
			то его нужно создать, так как от его имени будут создаваться курсы.
	\end{itemize}
\end{itemize}
\subsection{Настройка подсистемы сбора аналитики EDX}
Для запуска задач по расписанию, выполняющих сбор и агрегацию данных из БД \texttt{EDX}, хранилища курсов \texttt{EDX}, журнала событий \texttt{EDX} необходимо выполнить следующие подготовительные действия
\begin{itemize}
	\item С помощью менеджера конфигураций \texttt{ansible} выполнить установку роли \texttt{analytics-pipeline}
	\item Создать директорию для приложения сбора данных, например \texttt{/edx/app/pipeline}, перейти в созданную директорию
	\item Создать виртуальное окружение Python командой:\\
		\texttt{virtualenv pipeline}
	\item Активировать созданное виртуальное окружение командой:\\
		\texttt{source pipeline/bin/activate}
	\item Скопировать исходный код подсистемы сбора данных \texttt{edx-analytics-pipeline} из репозитория \url{https://github.com/npoed/edx-analytics-pipeline} в какой-либо каталог (создание каталога для файлов подсистемы может потребовать выполнения от администратора системы) и перейти в этот каталог в интерпретаторе командной строки
	\item Сформировать окружение для подсистемы сбора данных командой:\\
		\texttt{make bootstrap}
	\item Создать хранилище для подсистемы сбора данных командой:\\
	\begin{sloppypar}
		\texttt{sudo -Hu hadoop /edx/app/hadoop/hadoop/bin/hadoop fs -mkdir hdfs://127.0.0.1:9000/edx-analytics-pipeline/warehouse}		
	\end{sloppypar}
	\item Создать директорию временных выходных данных для подсистемы сбора данных командой:\\
	\begin{sloppypar}
		\texttt{sudo -Hu hadoop /edx/app/hadoop/hadoop/bin/hadoop fs -mkdir hdfs://127.0.0.1:9000/edx-analytics-pipeline/output\_root}		
	\end{sloppypar}
	\item Создать директорию для обработки журнала событий командой:\\
	\begin{sloppypar}
		\texttt{sudo -Hu hadoop /edx/app/hadoop/hadoop/bin/hadoop fs -mkdir hdfs://127.0.0.1:9000/data}
	\end{sloppypar}
	\item Поместить журнал событий в созданную директорию командой:\\
	\begin{sloppypar}
		\texttt{sudo -Hu hadoop /edx/app/hadoop/hadoop/bin/hdfs dfs -put -f /edx/var/log/tracking/tracking.log-* hdfs://localhost:9000/data}	
	\end{sloppypar}		
	\item Сформировать файл \texttt{/edx/etc/edx-analytics-pipeline/input.json} с параметрами доступа к БД \texttt{EDX} следующего содержания:
	\lstinputlisting[style=text]{src/input.json}	
	\item Сформировать файл \texttt{/edx/etc/edx-analytics-pipeline/output.json} с параметрами доступа к БД результатов сбора данных следующего содержания:
	\lstinputlisting[style=text]{src/output.json}
	\item Создать БД для метаданных подсистемы обработки событий командой:\\
		\texttt{mysql> CREATE DATABASE edx\_hive\_metastore CHARACTER SET = latin1;}	
	\item Выдать разрешения на БД для метаданных подсистемы обработки событий командой:\\
		\texttt{mysql> GRANT ALL PRIVILEGES ON edx\_hive\_metastore.* TO edx\_hive\_metastore@host IDENTIFIED BY 'edx';}			
	\item В файле \texttt{/edx/app/hadoop/hive/conf/hive-site.xml} указать параметры доступа к БД для метаданных подсистемы обработки событий. В результате должен получиться файл следующего содержания:\\
		\lstinputlisting[style=text]{src/hive-site.xml}
	\item В файлах \texttt{/edx/app/pipeline/edx-analytics-pipeline/client.cfg} и \texttt{/edx/app/pipeline/edx-analytics-pipeline/override.cfg} указать параметры:
	\begin{itemize}
		\item путь к исполняемому файлу интерпретатора \texttt{Python} в виртуальном окружении подсистемы обработки событий в секции \texttt{hadoop}
		\item доступ к БД результатов сбора данных в секции \texttt{database-export}
		\item доступ к БД \texttt{EDX} в секции \texttt{database-import}
		\item доступ к хранилищу курсов \texttt{EDX} в секции \texttt{mongodb}
		\item шаблон имени файла с событиями \texttt{EDX} в секции \texttt{event-logs}
	\end{itemize}
	 В результате должен получиться файл \texttt{client.cfg} следующего содержания:\\
		\lstinputlisting[style=text,breaklines=true]{src/client.cfg}	
	 В результате должен получиться файл \texttt{override.cfg} следующего содержания:\\		
		\lstinputlisting[style=text]{src/override.cfg}
\end{itemize}
После подготовительных действий необходимо добавить в планировщик задач одного их хостов, на котором исполняется \texttt{EDX}, директивы для выполнения от пользователя \texttt{hadoop} следующих команд один раз в сутки:
\begin{sloppypar}
\begin{itemize}
	\item \texttt{/edx/app/hadoop/hadoop/bin/hadoop fs -rm -r hdfs://127.0.0.1:9000/edx-analytics-pipeline/warehouse/*}
	\item \texttt{/edx/app/hadoop/hadoop/bin/hadoop fs -rm -r hdfs://127.0.0.1:9000/edx-analytics-pipeline/dest}
	\item \texttt{/edx/app/hadoop/hadoop/bin/hadoop fs -rm -r hdfs://127.0.0.1:9000/edx-analytics-pipeline/dest/*}
	\item \texttt{mysql> DROP DATABASE edx\_hive\_metastore;}
	\item \texttt{mysql> CREATE DATABASE edx\_hive\_metastore CHARACTER SET = latin1;}
	\item \texttt{/edx/app/pipeline/pipeline/bin/launch-task ActivityWorkflow -{}-local-scheduler -{}-n-reduce-tasks 4 -{}-interval 2016-04-01-2016-04-30 -{}-overwrite}
	\item \texttt{/edx/app/pipeline/pipeline/bin/launch-task CustomEnrollmentTaskWorkflow -{}-local-scheduler -{}-n-reduce-tasks 4 -{}-interval 2016-04-01-2016-04-30 -{}-overwrite}
	\item \texttt{/edx/app/pipeline/pipeline/bin/launch-task MongoImportTaskWorkflow -{}-local-scheduler -{}-overwrite}
\end{itemize}
\end{sloppypar}
Значение параметра \texttt{n-reduce-tasks} должно быть установлено равным количеству доступных ядер на хосте, на котором производится сбор аналитики \texttt{EDX}.

\subsection{Настройка подсистемы сбора аналитики PLP}
Для запуска задач по расписанию, выполняющих сбор и агрегацию данных из БД результатов сбора данных \texttt{EDX} (\texttt{reports}) необходимо проинициализировать представления и хранимые процедуры командой, выполненной из виртуального окружения \texttt{PLP}:\\
	\texttt{./manage.py init-analytics --settings=settings.local\_settings}\\
После подготовительных действий необходимо добавить в планировщик задач одного их хостов, на котором исполняется \texttt{PLP}, директиву для выполнения команды, приведённой ниже. Данная команда должна выполняться один раз в сутки от пользователя, от которого запускается \texttt{PLP}.\\
	\texttt{./manage import-analytics 2016-03-01 --settings=settings.local\_settings}\\
При запуске команды необходимо в качестве аргумента указывать дату, предшествующую дню, когда производится запуск. Формат даты задаётся в параметре \texttt{EVENT\_DATE\_FORMAT} в настройках \texttt{PLP}.

\section{Настройка PLP}
Для дальнейшей настройки \texttt{PLP} нужно:
\begin{itemize}
	\item Создать базу \texttt{MySQL} для \texttt{PLP}. Для этого необходимо выполнить следующую команду в консоли \texttt{MySQL}:\\
		\texttt{CREATE DATABASE <имя базы> CHAR SET \textquotesingle{}UTF8\textquotesingle{};}
	\item Скопировать файл \texttt{npoed\_plp/settings/develop.py} в\\ \texttt{npoed\_plp/settings/local\_settings.py}
	\item В \texttt{local\_settings.py} нужно заменить следующие значения:
	\begin{itemize}
		\item В \texttt{DATABASES[\textquotesingle{}default\textquotesingle{}]} нужно задать:
		\begin{itemize}
			\item \texttt{\textquotesingle{}ENGINE\textquotesingle{}:} \texttt{\textquotesingle{}django.db.backends.mysql\textquotesingle{}}
			\item \texttt{\textquotesingle{}NAME\textquotesingle{}:} имя созданной базы \texttt{MySQL}
			\item \texttt{\textquotesingle{}USER\textquotesingle{}:} имя пользователя для подключения к базе
			\item \texttt{\textquotesingle{}PASSWORD\textquotesingle{}:} пароль пользователя для подключения к базе
			\item \texttt{\textquotesingle{}HOST\textquotesingle{}:} адрес хоста, на котором расположена база
			\item \texttt{\textquotesingle{}PORT\textquotesingle{}:} порт, на котором запущен \texttt{MySQL}
		\end{itemize}
		\item \texttt{SSO\_NPOED\_URL} --- адрес, по которому развёрнуто \texttt{SSO}
		\item \texttt{SOCIAL\_AUTH\_NPOEDSSO\_KEY} --- \texttt{Client id}, заданный в \texttt{SSO} для \texttt{PLP}
		\item \texttt{SOCIAL\_AUTH\_NPOEDSSO\_SECRET} --- \texttt{Client secret}, заданный в \texttt{SSO} для \texttt{PLP}
		\item \texttt{SSO\_API\_TOKEN} --- токен суперпользователя, созданный в \texttt{SSO}
		\item \texttt{SSO\_API\_KEY} --- \texttt{API}-ключ \texttt{SSO}
		\item \texttt{EDX\_ENROLLMENT\_URL} --- адрес, по которому развёрнуто \texttt{EDX LMS}
		\item \texttt{EDX\_CMS\_URL} --- адрес, по которому развёрнуто \texttt{EDX CMS}
		\item \texttt{EDX\_API\_KEY} --- \texttt{API}-ключ \texttt{EDX}
		\item \texttt{SITE\_NAME} --- внешний адрес \texttt{PLP}
		\item \texttt{BROKER\_VHOST} --- имя виртуального хоста RabbitMQ
		\item \texttt{BROKER\_USER} --- имя пользователя RabbitMQ
		\item \texttt{BROKER\_PASSWORD} --- пароль пользователя RabbitMQ
	\end{itemize}
	\item Для подготовки базы данных к работе нужно выполнить следующие команды (виртуальное окружение должно быть активировано):\\
		\texttt{./manage.py makemigrations -{}-settings=settings.local\_settings}\\
		\texttt{./manage.py migrate -{}-settings=settings.local\_settings}
	\item Для загрузки начальных данных, необходимых для работы приложения, необходимо выполнить команду:\\
		\texttt{./manage.py load-init-data -{}-settings=settings.local\_settings}
	\item Необходимо убедиться в наличии прав на чтение в эту директорию, где расположены файлы \texttt{PLP} для пользователя, от которого будет запущено \texttt{PLP}		
	\item Необходимо убедиться в наличии директорий (\texttt{STATIC\_ROOT}, \texttt{MEDIA\_ROOT} и \texttt{CSV\_ROOT}), указанных в настройках,
		а также прав на чтение и запись в эти директории для пользователя, от которого будет запущено \texttt{PLP}
\end{itemize}

\section{Настройка запуска PLP}
Для запуска \texttt{PLP} необходимы следующие программы: \texttt{Supervisor}, \texttt{Nginx} и \texttt{gunicorn}.
Для их настройки нужно выполнить следующие действия:

\begin{itemize}
	\vbox{\item Создать conf-файл для \texttt{Supervisor} следующего содержания:
	\lstinputlisting[style=text]{src/plp.conf}}
	Если в настройках \texttt{PLP} задано логирование в консоль, то эти сообщения попадут
	в соответствующие лог-файлы \texttt{Supervisor} (указанные в conf-файле выше)
	\item Создать конфигурационный файл для \texttt{Nginx} следующего содержания:
		\lstinputlisting[style=text]{src/plp}
		В этом файле нужно задать адрес \texttt{PLP}, адрес и порт \texttt{gunicorn}, расположение сертификатов и расположение статических файлов.
	\item Создать конфигурационный файл для gunicorn, назвать его \texttt{plp\_gunicorn.py} и подкаталог \texttt{npoed\_plp} установленного приложения.
		Содержимое этого файла должно выглядеть примерно так:
		\lstinputlisting[style=text]{src/plp_gunicorn.py}
\end{itemize}
